\documentclass[10pt,a4paper]{article}
\usepackage{geometry}
\geometry{top=1in}
\usepackage{graphicx}
\usepackage{caption}
\usepackage{subcaption}

\title{Analysis of heuristic functions for game of Isolation}
\author{Omar A. Serrano}

\begin{document}

\maketitle

\begin{figure}
\centering
  \begin{subfigure}{.5\textwidth}
    \centering
    \includegraphics[width=\linewidth]{ID_Improved.png}
    \caption{ID\_Improved}
    \label{fig:sub2}
  \end{subfigure}%
  \begin{subfigure}{.5\textwidth}
    \centering
    \includegraphics[width=\linewidth]{heuristic_1.png}
    \caption{heuristic\_1}
    \label{fig:sub1}
  \end{subfigure}
  \begin{subfigure}{.5\textwidth}
    \centering
    \includegraphics[width=\linewidth]{heuristic_2.png}
    \caption{heuristic\_2}
    \label{fig:sub3}
  \end{subfigure}%
  \begin{subfigure}{.5\textwidth}
    \centering
    \includegraphics[width=\linewidth]{heuristic_3.png}
    \caption{heuristic\_3}
    \label{fig:sub4}
  \end{subfigure}
\label{fig:test}
\caption{Performance of heuristic functions vs. ID\_Improved}
\end{figure}

The three heuristic functions defined in \textit{game\_agent.py} are very
similar, because they use the same factors to evaluate the board. All three of
them compute the difference between the number of legal moves for each player,
which is essentially the evaluation function used by \textit{ID\_Improved}.
Additionally, all heuristic functions penalize locations that are farther away
from the center of the board. These two factors alone define the full set of
factors evaluated by \textit{heuristic\_1} and \textit{heuristic\_2}, whereas
\textit{heuristic\_3} also computes the difference between the number of
blank spaces within two spaces away of a player's location on the board.

All three heuristic functions evaluate the difference between the number of
legal moves for each player equivalently, but not all use the same approach to
score the distance from the center of the board. \textit{heuristic\_1}
multiplies the distance to the center by a weight that decreases as the game
progresses. The importance of being close to the center of the board is that
it may be easier to jump to different areas of the board. Thus,
\textit{heuristic\_1} and \textit{heuristic\_2} use this, but
\textit{heuristic\_1} makes it less important toward the end of the game, the
idea being there are less empty spaces by then, at which point it may be
advantageous to hover around the edges of the grid.

\textit{heuristic\_3} is the same as \textit{heuristic\_2}, except that it
adds the value of the difference between free spaces that are within two spaces
away from a player's location, multiplied by a weight that increases as the game
progresses, making the factor more important toward the end of the game. The
idea is that a player should prefer a move that leads to locations with a higher
density of free spaces, which point it may be advantageous to hover around the
edges of the grid, which would become more important toward the end of the game
because free spaces becomes a more limited resource.

As Figure 1 illustrates, all three heuristics outperformed
\textit{ID\_Improved} in a tournament of 1000 matches, but \textit{heuristic\_1}
is the only one that managed a substantial iprovement. For
\textit{heuristic\_3}, it seems like looking at the density of free spaces
around a player hurts the performance. This is not very intuitive, because it
seems reasonable to assume a higher number of free spaces would be correlated
with a player's ability to move, but perhaps it is due to the fact that agents
move like chess knights, for whom open positions are less important.
Clearly, giving less importance to being close to the center of the board as
the game progresses makes a significant difference, because this is the only
aspect that differentiates \textit{heuristic\_1} from \textit{heuristic\_2},
but it improves the winning rate by 3\%.

In conclusion, the results make it an easy choice to select
\textit{heuristic\_1} as the default heuristic function used by the Isolation
game agent.

\end{document}
